\documentclass[a4paper]{article}

\usepackage{ngerman}
\usepackage{a4wide}
\usepackage{graphicx}
\usepackage[utf8]{inputenc}
\usepackage{amscd}
\usepackage{amssymb}
\usepackage{amsmath}
\usepackage{bibgerm}

%%%
%%% Style-Definition des Literaturverzeichnis
%%%
%%\bibliographystyle{alpha}
\bibliographystyle{geralpha}
 
%%\setlength{\parindent}{0pt} %kein Einzug beim Absatzbegin
\setlength{\parskip}\medskipamount %Abstand zwischen 2 Abs�tzen

\author{Matthias Kemmer, Julius Hackel, Markus Bullmann, Stefan Gerasch}
\title{Gliederung Vertiefungsseminar MI}
\begin{document}

\maketitle
\tableofcontents

\section{Einführung}

\subsection{Prinzip}
\subsection{Beispiel (Sound \& Visualisierung)}
\subsection{Geschichte (auch GS1)}

\section{Technisches und Formeln}
\label{technisches}

\subsection{Grundlegende Erläuterungen}
\subsubsection{Auffrischung: Sinus- und Kosinusfunktion mit Parametern}
Die einfachste Form eines Tons lässt sich durch eine Sinus- oder Kosinusschwingung beschreiben. Dabei handelt es sich mathematisch gesehen um eine Sinus- oder eine Kosinusfunktion. Beide gehören zu den trigonometrischen Funktionen, auch Winkelfunktionen genannt. Damit einige später folgende mathematische und für die FM-Synthese erforderliche Berechnungen besser verstanden werden können, wird hier kurz auf die Grundlagen zu Sinus- und Kosinus Funktionen eingegangen. 

Sinus und Kosinus sind periodische Funktionen, d.h. die Funktionswerte wiederholen sich nach einer sogenannten Periode. Mathematisch ausgedrückt muss es dafür eine konstante p  geben, für die bei einem beliebigen x gilt: $f(x + p) =  f(x)$ .

Am besten verdeutlichen kann man dies anhand des Einheitskreises. Abbildung \ref{fig:unitcircle} zeigt, wie sich Sinus und Kosinus aus den Seiten eines rechtwinkligen Dreiecks im Einheitskreis berechnen lassen:

\begin{figure} [ht]
\centering
\includegraphics[width=0.95\textwidth]{Unit_Circle.png}
\caption{Der Einheitskreis}
\label{fig:unitcircle}
Quelle: \cite{fmtheory} - Fig. 2.6
\end{figure}

Dabei gilt: 

$\sin(\pi) = \frac{\text{gegenüberliegende Seite}}{\text{Hypotenuse}} = \frac{a}{c}$, wobei die Hypotenuse im Einheitskreis eine Länge von 1 hat. 

$\to\sin(\pi) = \frac{a}{1} = a$.

$\cos(\pi) = \frac{\text{anliegende Seite}}{\text{Hypotenuse}} = \frac{b}{c} = \frac{b}{1} = b$ \cite[s. 22 - 27]{fmtheory} \\

\begin{figure} [ht]
\centering
\includegraphics[width=0.95\textwidth]{sinus_einheitskreis.png}
\caption{Vom Einheitskreis zum Sinus}
\label{fig:unitcircleToSinus}
Quelle: \url{http://www.ulrich-rapp.de/stoff/mathematik/Sinus_Einheitskreis.gif}
\end{figure}

Der aktuelle Winkel im Einheitskreis kann auch im sogenannten Bogenmaß angegeben werden. Das Bogenmaß beschreibt, wie weit man den Bogen des Einheitskreises im Uhrzeigersinn abgelaufen ist. Dabei ist $\pi$ die sogenannte Kreiszahl und ein Bogenmaß von $2\pi$ gibt eine ganze Umdrehung im Einheitskreis an bzw. $\pi$ dann eine halbe Umdrehung. Um nun auf eine Sinusfunktion überzuleiten, muss das Bogenmaß in Abhängigkeit von der Zeit t angegeben werden, die Formel lautet also dann $\sin(2\pi*t)$. Damit festgelegt werden kann, wie oft der Einheitskreis in einer Sekunde abgelaufen wird, multipliziert man das Bogenmaß $2\pi$ mit dem Faktor $f$. Dabei entspricht $f$ der Frequenz der Funktion, also wie viele Schwingungen in einer Sekunde durchlaufen werden. Der Wert $2\pi * f$ wird auch als Kreisfrequenz bezeichnet.
Man spricht bei Sinus- und Kosinusfunktion auch von Winkelfunktionen, denn man kann den Funktionswert der Funktion auch anhand des Winkels (in Bogenmaß) im Einheitskreis angeben. 
Eine Veranschaulichung des Zusammenhangs zwischen dem Einheitskreis und der Sinusfunktion zeigt Abbildung \ref{fig:unitcircleToSinus}.

Der Unterschied von der Kosinusfunktion zur Sinusfunktion ist, dass der Kosinus bei dem Funktionswert 1 beginnt und Sinus bei Funktionswert 0. Aus diesem Zusammenhang lassen sich folgende Beziehungen (Auch: Komplementärformeln) zwischen Sinus und Kosinus feststellen:\\

\begin{lstlisting}[mathescape]
$\sin(\frac {\pi}{2} – x) =\cos(x)$
$\cos(\frac {\pi}{2} – x) = \sin(x)$ 
\end{lstlisting}\cite[s. 218]{matheBuch}

Überträgt man diese mathematischen Erkenntnisse nun auf einen Ton, so kann man dessen Funktion wie folgt beschreiben: 		$y(t) = y_0 * \sin(2 \pi f * t)$

Um 90\degree oder $\frac{\pi}{2}$ verschoben gilt: 		$y(t) = y_0*\cos(2 \pi f*t)$

Bei $y_0$ handelt es sich um die Amplitude des Tons, also dessen Lautstärke. Alle Funktionswerte der Funktion werden mit diesem Faktor multipliziert und dadurch größer oder kleiner.

F ist wie oben bereits beschrieben die Frequenz des Tons. Sie gibt die Anzahl der Schwingungen (Perioden) pro Sekunde an und wird in $f=\frac{1}{T}$ angegeben, wobei T die Periodendauer ist. Die Einheit ist Hertz(Hz).
Je größer die Frequenz eines Tones ist, desto höher klingt er für das Ohr.

Physikalisch gesehen ist ein Ton eine periodische Änderung des Luftdruckt, also Luftmoleküle die um ihre Ruhelage in Schwingung versetzt werden\cite[s. 111 f.]{zwicker}. 

Für das menschliche Ohr hören sich Sinus- und Cosinusschwingungen gleich an, da diese lediglich um $\frac{1}{4}$ der Periode verschoben sind und es in der realen Welt ohnehin keine perfekten Schwingungen geben kann \cite[s. 3f.]{zwicker}. 

\subsubsection{Parameter der FM-Synthese nach Chowning}

In diesem Kapitel wird auf die unterschiedlichen Parameter eingegangen, welche in der Formel der einfachen Frequenzmodulation von John Chowning vorkommen. Zudem wird deren Funktion im Einzelnen beschrieben.

Die Parameter werden mit der Gleichung in Sinus-Darstellung vorgestellt, d.h. Trägersignal und Modulator sind Sinusfunktionen. Dieselbe Erklärung funktioniert jedoch analog dazu auch mit Cosinus-Funktionen für den Träger und den Modulator.
Die Gleichung für eine frequenzmodulierte Welle lautet wie folgt:
\[ e(t) = A sin(\alpha t + I sin(\beta t)) \]

\begin{lstlisting}[mathescape]
- $ e(t) $ beschreibt die Amplitude des Modulierten Signals zum Zeitpunkt t
- A stellt die höchste Amplitude des modulierten Signals dar
- $ \alpha $ ist die Kreisfrequenz des Trägersignals in $\frac{1}{s}$
- $ \beta $ ist die Kreisfrequenz des Modulators in $\frac{1}{s}$
- I stellt den Modulationsindex dar. Dieser setzt sich wie folgt zusammen: $ I = \frac{d}{m} $ mit:
	= d: Frequenzhub der Modulation, d.h. der größte momentane Unterschied zwischen der Frequenz des Trägers und der des modulierten Signals
	Auch: Frequenzänderung, welche durch die Modulation der Trägerfrequenz verursacht wird
	= m: Kreisfrequenz des Modulators
\end{lstlisting} \cite{chowningPaper}

Der Modulationsindex beschreibt also das Verhältnis des Frequenzhubs zur Modulationsfrequenz.
Man kann anhand der Formel bereits erkennen, dass bei einem Modulationsindex von 0 keine Modulation stattfindet. Dabei wird der Frequenzhub der Modulation auch gleich 0, da $ I=\frac{d}{m} $ und somit $ d = 0*m = 0 $ gilt. 
\[ e(t) = A \sin(\alpha t + 0 \sin(\beta t))  \Rightarrow  e(t) = a \sin(\alpha t) \]
Die Frequenz in Hz (Schwingungen pro Sekunde) ergibt sich, wenn man die Kreisfrequenz (siehe oben) durch $2\pi$ teilt.
Anhand der Frequenz des Modulators kann man sehen, wie oft der oben beschriebene Frequenzhub pro Sekunde durchlaufen wird. Beispiel: Bei einer Modulationsfrequenz von 440 Hz wird der Frequenzhub 440 mal pro Sekunde durchlaufen.


\subsection{Besonderheiten der FM-Synthese}

\subsubsection{Phasenmodulation als FM}
Frequenzmodulation (FM) und Phasenmodulation (PM) können unter dem Oberbegriff Winkelmodulation zusammen gefasst werden. Wie der Name andeutet wird der Phasenwinkel eines Trägersignales in Abhängigkeit eines Modulationssignals verändert. In der allgemeinsten Form kann ein winkelmoduliertes Signal als Sinusfunktion eines sich zeitlich ändernden Winkels beschrieben werden.
\begin{equation}
s(t)=A*sin(\theta(t))
\end{equation}
Dabei wird $\theta(t)$ als \textit{momentaner Winkel} bezeichnet und ist als Summe der konstanten Kreisfrequenz $\omega_0$ multipliziert mit der Zeit $t$ und der \textit{momentanen Phase} $\varphi(t)$ definiert:
\begin{equation}
\theta(t)=\omega_0t + \varphi(t)
\end{equation}
Die \textit{momentan Frequenz} $\omega_m(t)$ entspricht der Änderung des Phasenwinkels in Abhängigkeit der Zeit. Daher kann die momentan Frequenz durch die erste Ableitung des Phasenwinkels nach der Zeit $\dot \theta(t)$ bestimmt werden.
\begin{equation}
\omega_m(t)=\dot \theta(t)=\frac{d\theta(t)}{dt}=\frac{d[\omega_0t+\varphi(t)]}{dt}=\frac{d\omega_0t}{dt}+\frac{d\varphi(t)}{dt}=\omega_0+\frac{d\varphi(t)}{dt}
\end{equation}


\textbf{ANALOGIE PHYSIK/BEGRÜNDUNG WARUM ABLEITUNG}

Wird nun die momentane Phase des Trägersignals proportional zum Modulationssignal $p(t)$ verändert, erhält man das \textbf{phasenmodulierte} Signal $s_{PM}(t)$.\\
Wird jedoch die momentante Frequenz des Trägersignals proportional zum Modulationssignal $f(t)$ verändert, erhält man das \textbf{frequenzmodulierte} Signal $s_{FM}(t)$.

Mit dieser Erkenntnis ergeben sich für die Phasenmodulation folgende mathematische Zusammenhänge:
\[
\varphi(t)=\varphi_0+k_{PM}*p(t)
\]
\[
\Rightarrow
\]

\subsubsection{Seitenfrequenzbänder (Evtl. Besselfunktion)}
Was sind SFB?
Warum sind sie wichtig? (Ton Charakter etc)
Warum treten sie auf?
Wie berechnet man sie (Fourier Analyse/Bessel Funktion)
\subsubsection{Harmonische Frequenzverhältnisse(Inkl. Vibrato)}

\section{Prinzipien der komplexen FM-Synthese}
\label{PrinzipKomplexFM}

Die bisherige Darstellung der Soundsynthese durch Frequenzmodulation (FM) hat sich auf die \textit{einfache} FM konzentriert, d.h. auf die Modulation eines einzelnen Trägers durch einen einzigen Modulator. Dadurch konnten die Prinzipien der Einflussnahme auf die Sounderzeugung über die Parameter von Trägerfrequenz, Modulationsfrequenz sowie den Modulationsindex analytisch klar isoliert werden. In der Praxis ist diese Limitation nicht notwendig, so dass zur Erzeugung komplexer Klangspektren auf den gleichzeitigen Einsatz mehrerer Träger und Modulatoren zurückgegriffen wird. Diese können dabei flexibel miteinander verschaltet werden. Man spricht in diesem Zusammenhang von der \textit{komplexen} FM. Im Folgenden werden die Prinzipien der komplexen FM näher beleuchtet. Auch hier wird zunächst mit einer künstlichen Limitation gearbeitet und der bisher betrachtete Fall entweder um einen weiteren Träger erweitert oder um einen weiteren Modulator. Denkbar sind unter diesen Bedingungen zwei grundsätzliche Verschaltungsmuster, und zwar zum einen eine Parallelschaltung, zum anderen eine Reihenschaltung. Als spezielle Form der Reihenschaltung kann die Rückkopplung des Trägersignals auf sich selbst gesehen werden; dieser Fall der Feedback-Schaltung wird ebenfalls vorgestellt. Anschließend wird am Beispiel des digitalen Synthesizers \textit{FM8} des deutschen Herstellers \textit{Native Instruments} beispielhaft gezeigt, inwiefern moderne Software die Klangerzeugung durch komplexe FM für den Benutzer handhabbar macht. 

\subsection{Parallelschaltung}

\subsubsection{Parallele Träger mit jeweils eigenem unabhängigen Modulator}

Zur Vollständigkeit soll die simpelste Erweiterung der einfachen FM nicht unerwähnt bleiben, und zwar die Situation zweier paralleler Träger, denen je ein eigener Modulator zugeordnet ist (Abb 1). Dabei handelt es sich streng genommen noch um keine komplexe FM, da die Signalpfade beider Träger erst auf der Ebene des Gesamtausgangssignals gemischt werden und sich ansonsten nicht weiter beeinflussen. Jedes Träger-Modulator-Paar, also $T1$ mit $M1$ sowie $T2$ mit $M2$, folgt dabei für sich genommen den Prinzipien der einfachen FM, so dass für jedes Paar die Lage der Seitenfrequenzbänder gesondert bestimmt und addiert werden kann, womit sich das Ausgangssignal $T1<M1 + T2<M2$ insgesamt ergibt. Lage und Amplitude der Seitenfrequenzbänder folgen direkt aus der gewählten Funktion für die Frequenzmodulation, auch wenn die Herleitung nichttrivial ist und Kenntnis der Eigenschaften der Besselfunktion voraussetzt. \\
Alle Spektren und Wellenformen in diesem Kapitel der Seminararbeit wurden auf Basis folgender Funktion für die FM modelliert. Auf einen Koeffizienten $A$, der üblicherweise für die Amplitude des gesamten Ausgangssignals vorne auf die Funktion aufmultipliziert wird, wird im Lauf der folgenden Ausführungen für verbesserte Übersichtlichkeit verzichtet:
\label{matze:simplefm}
\begin{equation} \label{eq:SimpleFM}
f_{FM}(t) = \sin(w_ct + I\sin(w_mt)) \quad \text{mit} \quad w_ct \quad \text{als Träger-} \quad \text{und} \quad w_mt \quad \text{als Modulationsfreq.}
\end{equation}
Um eine Funktion für das Frequenzspektrum zu erhalten, expandiert man diese über das trigonometrische Additionstheorem \begin{math} \sin(a + b) = \sin(a)\cos(b)+\cos(a)\sin(b) \end{math} und erhält
\begin{equation}\label{eq:BesselZwischenform}
\sin(w_ct + I\sin(w_mt)) = \sin(w_ct)\cos(I\sin(w_mt)) + \cos(w_ct)\sin(I\sin(w_mt)).
\end{equation}
Die inneren Terme \begin{math} \cos(I\sin(w_mt)) \end{math} und \begin{math} \sin(I\sin(w_mt)) \end{math} können nun über die Besselfunktion wie folgt ausgedrückt werden. Die Herleitung geschieht über die Erzeugerfunktion; siehe \cite[S.361, Satz~9.1.42 und Satz~9.1.43]{abramowitz}:
\begin{equation}\label{eq:Besselsin}
\sin(I\sin(w_mt)) = 2J_1(I)\sin(w_mt)+2J_3(I)\sin(3w_mt)+...+2J_{2n+1}(I)\sin((2n+1)w_mt)+...
\end{equation}
\begin{equation}\label{eq:Besselcos}
\cos(I\sin(w_mt)) = J_0(I)+2J_2(I)\cos(2w_mt)+...+2J_{2n}(I)\cos(2nw_mt)+...
\end{equation}
Man setzt dies in \ref{eq:BesselZwischenform} ein und expandiert alle auftretende Terme der Form \begin{math} \sin(w_ct)\cos(2nw_mt) \end{math} über das Additionstheorem \begin{math} \sin(a)\cos(b) = \frac{1}{2}\left(\sin(x+y)+\sin(x-y)\right) \end{math} sowie alle Terme der Form \begin{math} \cos(w_ct)\sin((2n+1)w_mt) \end{math} über  \begin{math} \cos(a)\sin(b) = \frac{1}{2}\left(\sin(x+y)-\sin(x-y)\right) \end{math}. Letztlich gilt: 
\begin{equation}
\begin{split}
\sin(w_ct + I\sin(w_mt)) \\ &\quad = J_0(I)\sin(w_ct) \\ &\quad + J_1(I)(\sin(w_ct + w_mt) - \sin(w_ct - w_mt)) \\ &\quad + J_2(I)(\sin(w_ct + 2w_mt)+\sin(w_ct-2w_mt)) \\ &\quad  + J_3(I)(\sin(w_ct + 3w_mt) - \sin(w_ct - 3w_mt)) \\ &\quad  + ...
\end{split}
\end{equation}
(\cite[S.528]{chowningPaper}). Zieht man die Besselkoeffizienten in die Klammer, lässt sich die Lage der Seitenfrequenzbänder bereits aus der Funktion auslesen:
\begin{equation}\label{eq:FormelinchenBessel}
\begin{split}
\sin(w_ct + I\sin(w_mt)) \\ &\quad = J_0(I)\sin(w_ct) \\ &\quad + J_1(I)\sin(w_ct + w_mt) \quad\enspace - J_1(I)\sin(w_ct - w_mt) \\ &\quad + J_2(I)\sin(w_ct + 2w_mt) \quad + J_2(I)\sin(w_ct-2w_mt) \\ &\quad  + J_3(I)\sin(w_ct + 3w_mt) \quad - J_3(I)\sin(w_ct - 3w_mt) \\ &\quad  + ...
\end{split}
\end{equation}
Die erste Zeile zeigt hier die Lage der Trägerfrequenz an, während jede folgende Zeile stets die Lage eines oberen Seitenbands angibt (erster Term) sowie das dazugehörige untere Seitenband (zweiter Term). \\ 
Man kann in dieser Darstellung am Vorzeichen des jeweiligen Terms ablesen, dass Besselkoeffizienten ungerader Ordnung zu negativen Amplitudenwerten des jeweiligen unteren Seitenbandes führen. Da negative Frequenzen einer Phasenverschiebung um 180 Grad entsprechen, erhält man die endgültigen, akustisch wahrgenommenen Frequenzen durch die Spiegelung der negativen Frequenzen je einmal an Abszisse und Ordinate. Dies enspricht einer Phasenverschiebung (Spiegelung Abszisse) im positiven Bereich (Spiegelung Ordinate). Dabei kann es natürlich zur Auslöschung oder Verstärkung von Bändern durch die Addition ursprünglich negativer Frequenzbänder auf bereits vorhandene im positiven Bereich kommen. \\
Die Formel \ref{eq:FormelinchenBessel} kann übersichtlich zusammengefasst werden (\cite{schottstaedtWeb}) zu 
\begin{equation}\label{esq:Besselbabymonster}
\sin(w_ct + I\sin(w_mt)) = \sum_{n=-\infty}^{\infty}J_n(I)\sin(w_ct+nw_mt)
\end{equation}
Das Frequenzspektrum für die einfache FM weist nun im positiven als auch im negativen Bereich stets im Abstand \begin{math} w_m \end{math} um die Trägerfrequenz \begin{math} w_c \end{math} Seitenbänder auf. Die jeweiligen Amplituden werden dabei durch die Ordnung der Besselfunktion bestimmt in Abhängigkeit vom Modulationsindex, welcher als Argument an die Besselfunktion übergeben wird. Die Besselfunktion liefert dabei ab einem Modulationsindex größer ca. 2,5 auch negative Vorzeichen für die Seitenfrequenzen zurück, wie folgende grafischen Darstellung der Besselfunktion nach den Parametern Modulationsindex und Nummer des Seitenbands zeigt: \ref{fig:bessel3D}. \\
Die für beide Paare $T1<M1 + T2<M2$ so ermittelten Frequenzspektren können für diese Form der Parallelschaltung in einem abschließenden Schritt einfach addiert werden und liefern so das gesamte Frequenzspektrum des gemischten Ausgangssignals. \\
Dies lässt sich in MATLAB komfortabel visualisieren. Zunächst das erste Paar T1 und M1:
\FloatBarrier
\begin{figure} [ht]
\centering
  \includegraphics[width=0.95\textwidth]{parT1M1.png}
\caption{Einfache FM mit $f_c = 220$, $w_m = 440$, $I = 1$. }
Quelle: Eigene Darstellung in MATLAB
\end{figure}
\FloatBarrier
Hier das Spektrum für T2 und M2:
\FloatBarrier
\begin{figure} [ht]
\centering
  \includegraphics[width=0.95\textwidth]{parT2M2.png}
\caption{Einfache FM mit $f_c = 330$, $w_m = 550$, $I = 2$. }
Quelle: Eigene Darstellung in MATLAB
\end{figure}
\FloatBarrier
Bei den Spektren für beide einfachen FM sieht man sehr gut, dass sich die Seitenfrequenzbänder in Abständen der Modulationsfrequenz um die Trägerfrequenz nach beiden Seiten hin ausbreiten.\\
Und hier nun das Spektrum für die Parallelschaltung beider Träger-Modulator-Paare. Man sieht sehr gut, dass sich die Einzelspektren der aus den beiden Plots oben aufaddieren: 
\FloatBarrier
\begin{figure} [ht]
\centering
  \includegraphics[width=0.95\textwidth]{parT1M1T2M2.png}
\caption{Parallele FM mit $f_{c1} = 220$, $f_{c2} = 330$, $w_{m1} = 440$, $w_{m2} = 550$, $I_1 = 1$, $I_2 = 2$. }
Quelle: Eigene Darstellung in MATLAB
\end{figure}
\FloatBarrier

\subsubsection{Parallele Träger mit einem gemeinsamen Modulator}

Teilen sich zwei Träger $T1$ und $T2$ denselben Modulator $M$, kann das Frequenzspektrum ebenfalls als Summe zweier unabhängiger einfacher FM betrachtet werden. Hierzu berechnet man zunächst das Frequenzspektrum des einen Trägers $T1$ anhand Modulationsfrequenz und dem -index des Modulators M. Anschließend berechnet man das Frequenzspektrum für den zweiten Träger $T2$, wobei wieder Frequenz und Modulationsindex von Modulator $M$ bezogen werden. Aufsummiert ergibt sich das Gesamtspektrum. Das Interessante an dieser Form der Parallelschaltung besteht in der Möglichkeit, nur einen einzigen Modulator zur Modulation mehrerer vorhandener Träger, deren Modulatoren dieselbe Modulationsfrequenz aufweisen, zu verwenden. Dadurch können die freigewordenen Modulatoren für andere Zwecke verwendet werden, z.B. um einen der Träger durch zwei Modulatoren zu modulieren (siehe \ref{singlecarryparallelmod}) oder sogar zum Modulieren des bereits eingesetzten Modulators (was dann der Kaskadenschaltung entspräche, siehe \ref{cascade}). Die Möglichkeit, denselben Modulator für verschiedene Träger zu verwenden, bedeutet logischerweise, dass die \textit{individuelle} Modulierbarkeit beider Träger aufgegeben wird, da sich Änderungen an den Einstellungen von $M$ nun zwangsläufig sowohl auf $T1$ als auch auf $T2$ auswirken. Die gleiche Mächtigkeit zweier paralleler Träger mit demselben Modulator $M$ im Vergleich zum Setup mit mehreren Trägern, welche jeweils ihren eigenen Modulator mit denselben Einstellungen wie $M$ besitzen, sei im Folgenden illustriert. \\
Zunächst das Spektrum zweier Träger, die jeweils von eigenen Modulatoren mit gleichen Einstellungen moduliert werden. Zuerst für den einen Träger:
\FloatBarrier
\begin{figure} [ht]
\centering
  \includegraphics[width=0.95\textwidth]{parT1M1.png}
\caption{Einfache FM mit $f_c = 220$, $w_m = 440$, $I = 1$. }
Quelle: Eigene Darstellung in MATLAB
\end{figure}
\FloatBarrier
Nun für den zweiten Träger:
\FloatBarrier
\begin{figure} [ht]
\centering
  \includegraphics[width=0.95\textwidth]{parT2M.png}
\caption{Einfache FM mit $f_c = 550$, $w_m = 440$, $I = 1$. }
Quelle: Eigene Darstellung in MATLAB
\end{figure}
\FloatBarrier
Die parallele Modulation von T1 durch M1 sowie von T2 durch M2 wobei M1 und M2 exakt dieselben Einstellungen besitzen:
\FloatBarrier
\begin{figure} [ht]
\centering
  \includegraphics[width=0.95\textwidth]{parT1T2MM.png}
\caption{Parallele FM mit $f_{c1} = 220$, $f_{c2} = 550$, $w_{m1} = 440$, $w_{m2} = 440$, $I_1 = 1$, $I_2 = 1$. }
Quelle: Eigene Darstellung in MATLAB
\end{figure}
\FloatBarrier
Und abschließend zum Vergleich Wellenform und Frequenzspektrum für zweier Träger, die sich denselben Modulator teilen:
\FloatBarrier
\begin{figure} [ht]
\centering
  \includegraphics[width=0.95\textwidth]{parT1T2SameM.png}
\caption{Parallele FM mit $f_{c1} = 220$, $f_{c2} = 330$, $w_{m1} = 440$, $w_{m2} = 550$, $I_1 = 1$, $I_2 = 2$.  }
Quelle: Eigene Darstellung in MATLAB
\end{figure}
\FloatBarrier
Wie man sieht, sind die Spektren für die Modulation durch zwei verschiedene Modulatoren absolut identisch mit der Modulation durch denselben Modulator, sofern alle Modulatoren dieselben Einstellungen aufweisen.

\subsubsection{Einzelner Träger mit parallel geschaltetem Modulatorenpaar}
\label{singlecarryparallelmod}
Wird ein einzelner Träger T von zwei parallel geschalteten Modulatoren M1 und M2 moduliert, kann das erzeugte Frequenzspektrum nicht mehr einfach als Summe der Spektren zweier einfacher FM begriffen werden, da sich die Auswirkungen der beiden Modulatoren auf den Träger gegenseitig beeinflussen und zahlreiche \textit{Kombinationsseitenfrequenzen} (CHOWNING)oder \textit{Intermodulationsfrequenzen]} (SCHOTTSTAEDT) entstehen. Das ist bereits aus der Formel für diese Spielart des FM direkt ersichtlich, die analog zu \ref{eq:SimpleFM} wie folgt nach \cite[S.46]{schottstaedt}lautet:
\begin{equation}
f_{FMparallel}(t) = \sin(w_ct + I_1\sin(w_{m1}t) + I\sin(w_{m2}t))
\end{equation}
Nach mehrmaliger Anwendung der trigonometrischen Additionstheoreme zeigt sich im Vergleich mit \ref{eq:BesselZwischenform} bereits, dass diese Formel eine Vielzahl zusätzlicher Terme mit sich bringt:
\begin{equation}
\begin{split}
\sin(w_ct + I_1\sin(w_{m1}t) + I\sin(w_{m2}t)) \\ &\quad = \sin(w_ct)\cos(I_1\sin(w_{m1}t))\cos(I\sin(w_{m2}t)) \\ &\quad + \cos(w_ct)\sin(I_1\sin(w_{m1}t))\cos(I\sin(w_{m2}t)) \\ &\quad +\cos(w_ct)\cos(I_1\sin(w_{m1}t))\sin(I\sin(w_{m2}t)) \\ &\quad -\sin(w_ct)\sin(I_1\sin(w_{m1}t))\sin(I\sin(w_{m2}t))
\end{split}
\end{equation}
Substituiert man auch hier jedes \begin{math} \sin(I\sin(w_{mx}t) \end{math} mit \ref{eq:Besselsin} und \begin{math} \cos(I\sin(w_{mx}t)) \end{math}) mit \ref{eq:Besselcos} und wendet erneut die trigonometrischen Additionstheoreme an, so ergibt sich übersichtlich:
\begin{equation}\label{eq:ParallelKompakt}
\sin(w_ct + I_1\sin(w_{m1}t) + I\sin(w_{m2}t)) = \sum_{i=-\infty}^{\infty}\sum_{k=-\infty}^{\infty}J_i(I_1)J_k(I_2)\sin(w_ct + iw_{m1}t + kw_{m2}t)
\end{equation}
Das bedeutet, dass bereits für alle Kombinationen der Indizes von -2 bis +2 ganze 25 verschiedene Terme (16 für die Kombinationsmöglichkeiten von 1 und 2 sowie den Vorzeichen + und -; weitere 9 für die Kombinationen von 1 und 0 sowie 2 und 0 inklusive der beiden Vorzeichen sowie die Kombination von 0 mit sich selbst) und somit individuelle Frequenzbänder entstehen. Die entsprechenden Additionsterme sind wie folgt:
\begin{equation}\label{eq:wallofequations}
\begin{split}
\sin&(w_ct + I_1\sin(w_{m1}t) + I\sin(w_{m2}t)) \\
&= J_0(I_1)J_0(I_2)\sin(w_ct) \\
&+ J_0(I_1)J_1(I_2)\sin(w_ct + w_{m2}t) - J_0(I_1)J_1(I_2)\sin(w_ct - w_{m2}t) \\
&+ J_0(I_1)J_2(I_2)\sin(w_ct + 2w_{m2}t) + J_0(I_1)J_1(I_2)\sin(w_ct - 2w_{m2}t) \\
&+ J_1(I_1)J_0(I_2)\sin(w_ct + w_{m1}t) - J_1(I_1)J_0(I_2)\sin(w_ct - w_{m1}t) \\
&+ J_1(I_1)J_1(I_2)\sin(w_ct + w_{m1}t + w_{m2}t) - J_1(I_1)J_1(I_2)\sin(w_ct + w_{m1}t - w_{m2}t) \\
&- J_1(I_1)J_1(I_2)\sin(w_ct - w_{m1}t + w_{m2}t) + J_1(I_1)J_1(I_2)\sin(w_ct - w_{m1}t - w_{m2}t) \\
&+ J_1(I_1)J_2(I_2)\sin(w_ct + w_{m1}t + 2w_{m2}t) + J_1(I_1)J_2(I_2)\sin(w_ct + w_{m1}t - 2w_{m2}t) \\
&- J_1(I_1)J_2(I_2)\sin(w_ct - w_{m1}t + 2w_{m2}t) - J_1(I_1)J_2(I_2)\sin(w_ct - w_{m1}t - 2w_{m2}t) \\
&+ J_2(I_1)J_0(I_2)\sin(w_ct + 2w_{m1}t) + J_2(I_1)J_0(I_2)\sin(w_ct - 2w_{m1}t) \\
&+ J_2(I_1)J_1(I_2)\sin(w_ct + 2w_{m1}t + w_{m2}t) + J_2(I_1)J_1(I_2)\sin(w_ct + 2w_{m1}t - w_{m2}t) \\
&+ J_2(I_1)J_1(I_2)\sin(w_ct - 2w_{m1}t + w_{m2}t) - J_2(I_1)J_1(I_2)\sin(w_ct - 2w_{m1}t - w_{m2}t) \\
&+ J_2(I_1)J_2(I_2)\sin(2w_ct + 2w_{m1}t + 2w_{m2}t) + J_2(I_1)J_2(I_2)\sin(2w_ct + 2w_{m1}t - 2w_{m2}t) \\
&+ J_2(I_1)J_2(I_2)\sin(2w_ct - 2w_{m1}t + 2w_{m2}t) + J_2(I_1)J_2(I_2)\sin(2w_ct - 2w_{m1}t - 2w_{m2}t) \\
&+ ... 
\end{split}
\end{equation}
Sofern zwei Zeilen zur selben Kombination der Ordnungen der Besselfunktion gegeben sind, steht die ober Zeile für das obere Seitenfrequenzband, die Zeile darunter dem unteren Seitenfrequenzband, zumindest sofern die Modulationsfrequenz von M1 größer ist, als jene von M2 - in diesem Fall sind die linken Terme beider Zeilen das obere Frequenzband und die rechten das untere. Es lässt sich somit bereits durch nur einen Träger und zwei unabhängige Modulatoren eine gewaltige Fülle von Frequenzspektren erzeugen. Eine Abkürzung, um auf die Additionsterme für das Spektrum zu gelangen, sei hier noch kurz präsentiert: Sieht man sich z.B. die beiden Zeilen mit den Besselkoeffizienten \begin{math} J_1J_2 \end{math} an, so kann man sich vorstellen, dass diese beiden Zeilen zusammen die Kombinationen (1,2),(1,-2),(-1,2),(-1,-2) abdecken. Nun weiß man bereits, dass man 4 Terme braucht. Innerhalb jedes einzelnen Terms bekommt jeweils \begin{math} w_{m1} \end{math} als Koeffizient die erste Vektorkomponente, \begin{math} w_{m2} \end{math} die zweite Vektorkomponente. Für den Vektor (1,-2) lautet der Term somit \begin{math} J_1(I_1)J_2(I_2)\sin(w_ct + w_{m1}t - 2w_{m2}t) \end{math}. Der gesamte Term bekommt als Vorzeichen noch das Produkt der Vorzeichen der Vektorkomponenten jeweils potenziert mit dem Betrag der jeweiligen Komponente. Für das gewählte Beispiel wäre das folglich \begin{math} (1)^1*(-1)^2 = 1 \end{math}; der Term bekommt also ein positives Vorzeichen. Die Bedingung des Potenzierens der Vorzeichen leitet sich dabei aus folgender Eigenschaft der Besselfunktion ab, deren Beweis hier (KREH) nachgelesen werden kann:
\begin{equation}
J_{-n}(x) = (-1)^nJ_n(x)
\end{equation}
Im folgenden finden sich die Spektren desselben Trägers, einmal moduliert durch einen Modulator mit Einstellungen blablubb,
einmal moduliert durch einen Modulator mit Einstellungen onkonk.

Zum Vergleich die Anwendung beider Modulatoren mit unveränderten Einstellungen, wieder auf denselben Träger. Es fällt auf, dass das komplexe Spektrum dieser Form der Parallelschaltung die individuellen Beiträge beider Modulatoren kaum noch erkennen lässt:


\subsection{Kaskadenschaltung: Einzelträger mit Modulatorenpaar in Reihe}
\label{cascade}

Die bisher betrachteten Schaltungen war gemein, dass unabhängig davon, ob ein Träger, mehrere Träger, ein Modulator oder mehrere Modulatoren verwendet worden sind, in keinem der Fälle ein Modulator selbst durch einen weiteren Modulator moduliert worden ist. Diese Kombinationsmöglichkeit eines Trägers sowie zweier Modulatoren stellt jedoch ein weiteres mächtiges Instrument der FM dar und soll im folgenden beschrieben werden. Die Modulation eines Trägers durch einen Modulator, der selbst wiederum durch einen weiteren Modulator moduliert wird, also die Reihenschaltung T<-M1<-M2 lässt sich in die bisher gewählte Formel für die einfache FM,
\begin{equation}\label{eq:fmsimplex}
f_{FM}(t) = \sin(w_ct + I\sin(w_mt))
\end{equation}
wenig überraschend wie folgt integrieren (\cite[S.48]{schottstaedt}):
\begin{equation}\label{eq:fmkaskade}
f_{FMkaskade}(t) = \sin(w_ct + I_1\sin(w_{m1}t + I_2\sin(w_{m2}t)))
\end{equation}
Bereits an dieser Funktion erkennt man den verschachtelten Charakter der Kaskadenschaltung: Der Modulationswert $ I\sin(w_mt) $ wird bei der Kaskadenschaltung durch den Wert einer weiteren, kompletten Frequenzmodulation $ I_2\sin(w_{m1}t + I_2\sin(w_{m2}t)) $ ersetzt. Das Frequenzspektrum berechnet sich dabei wie folgt:
\begin{equation}
\sin(w_ct + I_1\sin(w_{m1}t + I_2\sin(w_{m2}t))) = \sum_{n=-\infty}^{\infty}\sum_{k=-\infty}^{\infty}J_i(I_1)J_k(nI_2)\sin(w_ct + nw_{m1}t + kw_{m2}t)
\end{equation}
Für die Herleitung brauchen keine ausschweifenden Umformungen durch Additionstheoreme mehr vorgenommen oder Eigenschaften der Besselfunktion ausgenutzt werden; hat man \ref{eq:fmsimplex}, \ref{eq:fmkaskade} sowie \ref{eq:freqfm} bereits bestimmt, so kann in der Formel für das Frequenzspektrum zu \ref{eq:fmsimplex}, also in
\begin{equation}\label{eq:freqfm}
\sin(w_ct + I\sin(w_mt)) = \sum_{n=-\infty}^{\infty}J_n(I)\sin(w_ct+nw_mt)
\end{equation}
einfach $ \sin(w_mt) $ durch $ \sin(w_{m1}t + I_2\sin(w_{m2}t)) $ ersetzt werden. Eine Reihe einfacher Umformungen führt dann zur Formel für das Frequenzsspektrum der Kaskadenschaltung:
\begin{equation}
\begin{split}
& \quad \sum_{n=-\infty}^{\infty}J_n(I_1)sin(w_{c}t + nI_2(sin(w_{m2}t))) \\
&= \sum_{n=-\infty}^{\infty}J_n(I_1)sin(w_{c}t+n(w_{m1}t + I_2(sin(w_{m2}t)))) \\
&= \sum_{n=-\infty}^{\infty}J_n(I_1)sin((w_{c}t+nw_{m1}t) + (I_2n(sin(w_{m2}t))))
\end{split}
\end{equation}
Da hier die beiden Klammern im Sinus exakt einer FM entsprechen, kann dort die Formel für ihr Spektrum \ref{eq:freqfm} substituiert werden. Es ergibt sich:
\begin{equation}
\sum_{n=-\infty}^{\infty}\sum_{n=-\infty}^{\infty}J_n(I_1)J_n(nI_2)sin(w_{c}t+nw_{m1}t + kw_{m2}t)
\end{equation}
Damit entspricht die Formel für das Frequenzspektrum der Kaskadenschaltung exakt jener für die Parallelschaltung \ref{eq:ParallelKompakt}, mit einem kleinen Zusatz: Es wird bei der Kaskadenschaltung zusätzlich der Modulationsindex $ I_2 $ des äußersten Modulators M2 der Reihenschaltung vor der Auswertung durch die Besselfunktion mit der Ordnung $ n $ des ersten Besselkoeffizienten multipliziert. Damit ändern sich auch bei den Additionstermen für das Spektrum im Vergleich zu jenen der Parallelschaltung \ref{eq:wallofequations} lediglich die Koeffizienten:
\begin{equation}
\begin{split}
\sin&(w_ct + I_1\sin(w_{m1}t) + I\sin(w_{m2}t)) \\
&= J_0(I_1)J_0(\mathbf{0}I_2)\sin(w_ct) \\
&+ J_0(I_1)J_1(\mathbf{0}I_2)\sin(w_ct + w_{m2}t) - J_0(I_1)J_1(I_2)\sin(w_ct - w_{m2}t) \\
&+ J_0(I_1)J_2(\mathbf{0}I_2)\sin(w_ct + 2w_{m2}t) + J_0(I_1)J_1(I_2)\sin(w_ct - 2w_{m2}t) \\
&+ J_1(I_1)J_0(\mathbf{1}I_2)\sin(w_ct + w_{m1}t) - J_1(I_1)J_0(I_2)\sin(w_ct - w_{m1}t) \\
&+ J_1(I_1)J_1(\mathbf{1}I_2)\sin(w_ct + w_{m1}t + w_{m2}t) - J_1(I_1)J_1(I_2)\sin(w_ct + w_{m1}t - w_{m2}t) \\
&- J_1(I_1)J_1(\mathbf{1}I_2)\sin(w_ct - w_{m1}t + w_{m2}t) + J_1(I_1)J_1(I_2)\sin(w_ct - w_{m1}t - w_{m2}t) \\
&+ J_1(I_1)J_2(\mathbf{1}I_2)\sin(w_ct + w_{m1}t + 2w_{m2}t) + J_1(I_1)J_2(I_2)\sin(w_ct + w_{m1}t - 2w_{m2}t) \\
&- J_1(I_1)J_2(\mathbf{1}I_2)\sin(w_ct - w_{m1}t + 2w_{m2}t) - J_1(I_1)J_2(I_2)\sin(w_ct - w_{m1}t - 2w_{m2}t) \\
&+ J_2(I_1)J_0(\mathbf{2}I_2)\sin(w_ct + 2w_{m1}t) + J_2(I_1)J_0(I_2)\sin(w_ct - 2w_{m1}t) \\
&+ J_2(I_1)J_1(\mathbf{2}I_2)\sin(w_ct + 2w_{m1}t + w_{m2}t) + J_2(I_1)J_1(I_2)\sin(w_ct + 2w_{m1}t - w_{m2}t) \\
&+ J_2(I_1)J_1(\mathbf{2}I_2)\sin(w_ct - 2w_{m1}t + w_{m2}t) - J_2(I_1)J_1(I_2)\sin(w_ct - 2w_{m1}t - w_{m2}t) \\
&+ J_2(I_1)J_2(\mathbf{2}I_2)\sin(2w_ct + 2w_{m1}t + 2w_{m2}t) + J_2(I_1)J_2(I_2)\sin(2w_ct + 2w_{m1}t - 2w_{m2}t) \\
&+ J_2(I_1)J_2(\mathbf{2}I_2)\sin(2w_ct - 2w_{m1}t + 2w_{m2}t) + J_2(I_1)J_2(I_2)\sin(2w_ct - 2w_{m1}t - 2w_{m2}t) \\
&+ ... 
\end{split}
\end{equation}
Man sieht sofort, dass damit die Trägerfrequenz ausschließlich durch $ I_1 $ definiert wird, da $ I_2 $ bei der Trägerfrequenz mit 0 multipliziert wird und die Besselfunktion der nullter Ordnung für das Argument 0 einfach den Wert 1 zurückgibt. Zweitens müssen sich für Besselkoeffizienten höherer Ordnung der Modulationsindex für M2, also $ I_2 $, und damit auch die Kombinationsseitenfrequenzen höherer Ordnung im Vergleich zur Parallelmodulation stärker auf das resultierende Spektrum auswirken. Drittens entfallen die Nicht-Kombinationsseitenfrequenzen vollständig, da die erwähnte Multiplikation mit 0 als Argument für die Besselfunktionen höherer Ordnung als Null den Wert 0 zurückliefern (siehe \ref{fig:Bessel2D}). Visualisiert in MATLAB zeigt sich, dass erwartungsgemäß um jedes Seitenband von T und M1 sich die Seitenbänder von M1 und M2 ausbreiten. Hier zunächst das Spektrum von T moduliert von M1:

Nun das Spektrum von M1 moduliert durch M2 (zum nachstellen wurde ein Träger mit den Einstellungen von M1 moduliert durch einen Modulator mit den Einstellungen von M2):

Und nun das Gesamtspektrum der Kaskadenschaltung:

\subsection{Feedbackschaltung}

Eine Spezialform der komplexen FM besteht darin, das Ausgangssignal des Trägers zur Modulation des Trägers heranzuziehen anstelle eines Modulators wie in der einfachen FM. Die Formel für eine Feedbackschaltung kann rekursiv wie folgt angegeben werden:
\begin{equation}
f_{FMfeedback}(t_{n}) = sin(w_{c}t_{n} + If_{FMfeedback}(t_{n-1})) \quad \text{mit} \quad f_{FMfeedback}(t_{-1}) = 0
\end{equation}
In MATLAB oder in einer beliebigen Programmiersprache kann diese Formel dann natürlich komfortabel iterativ oder rekursiv implementiert werden.
Das Frequenzspektrum berechnet sich dabei nach:
\begin{equation}
f_{FMfeedback}(t) = \sum_{n=1}^{\infty}\frac{2}{nI}J_n(nI)\sin{nw_{c}t}
\end{equation}
Man sieht sofort, dass die Einflussnahme auf die Feedbackschaltung durch den Modulationsindex, d.h. die Amplitude, mit man dem Ausgangssignal der Schaltung erlaubt, auf sich selbst zurückzuwirken, lediglich die Energie und die Steilheit des Abfallens der Energie von den näher am Träger liegenden Seitenbändern zu den weiter außen liegenden beeinflusst. Die Variable $n$ hängt jedoch nur von der Ordnung der Besselfunktion ab und durchläuft in jedem Fall ganzzahlig die Werte von 1 bis $\infty$. In jedem Fall ist die Lage der Seitenfrequenzbänder also immer ein ganzzahliges Vielfaches der Trägerfrequenz. Der Amplitudenabfall von Band zu Band erfolgt dabei bei konstantem Modulationsindex, und wenn man von der Einflussnahme durch $n$ auf den Wert der Besselfunktion absieht im Verhältnis $\frac{1}{2}$ zu $\frac{1}{3}$ zu $\frac{1}{4}$ zu $\frac{1}{5}$ etc. und weist damit Ähnlichkeit zum Spektrum einer Sägezahnwelle auf. Feedback-FM bietet damit die Möglichkeit, mit nur einem Operator Töne zu erzeugen, die einen rauen Charakter aufweisen. Im folgenden sehen wir das Spektrum einer einfachen FM eines Trägers mit den Einstellungen blablubb, moduliert durch einen Modulator mit denselben Einstellungen:

Im Vergleich hier das Spektrum der Feedbackschaltung eines Trägers rückgekoppelt auf sich selbst mit den Einstellungen wie der Träger oben:

Die Ähnlichkeit ist erwartungsgemäß recht hoch.

\subsection{Komplexe Soundsynthese mit Native Instruments' \textit{FM8}}

\subsubsection{Die FM-Matrix 1,5 S}


\subsubsection{ADSR-Hüllkurven-Generator 0,5 S}


\subsection{Praktische Anwendung der FM-Synthese}
\subsubsection{Nachbildung eines Instruments}
Da es bei der FM Synthese nicht möglich ist im vorfeld zu wissen was für ein Signal herauskommt. Ist es schwer mit dieser Technik ein echt wirkendes Instrument nachzubilden.
Trotzdem gibt es einige Methoden den generierten Klang natürlicher wirken zu lassen. Diese werden im weiteren Verlauf diese Kapitels vorgestellt und anschließend versucht den Klang eines Instruments zu erzeugen.

Hüllkurven
Da bei vielen Instrumenten die Lautstärke während der Laufzeit eines Tones variiert, und ein abruptes Ein oder Ausschalten des Tones nicht besonders gut klingt, ist die nutzung einer Hüllkurve oder ADSR-Hüllkurve ein wichtiger Bestandteil der Nachbildung eines Instrumentes. ADSR steht für die einzelnen Phasen eines Tons: Attack, Decay, Sustain und Release. Diese Phasen sollen hier vereinfacht erklärt werden. Beim Drücken einer Taste wird der Ton angeschlagen und die Lautstärke des Tons steigt schnell bis zu einem maximal Wert an. Diese Phase wird Attack-Phase genannt. Nachdem die maximale Lautstärke erreicht wurde, startet die Decay Phase. In dieser Phase sinkt die Lautstärke schnell auf einen geringeren Wert ab. Danach befindet sich der Ton in der Sustain Phase und die Lautstärke bleibt gleich, solange der Ton gespielt wird. Sobald die Taste losgelassen wird, nimmt die Lautstärke wieder bis zu ihrem minimal Wert ab. 

In Abb. 1 ist der Verlauf der Lautstärke einer Standard ADSR-Hüllkurve noch einmal grafisch dargestellt.

Da allerdings bei vielen Instrumenten die Lautstärke in den einzelnen Phasen der ADSR-Hüllkurve nicht gleichmäßig steigt oder sinkt, ist es nötig die Kurven zu variieren. Zum Biespiel steigt bei vielen Instrumenten die Lautstärke in der Attack Phase exponentiell an und fällt in der Decay und Release Phase auch exponentiell ab. Manche Synthesizer bieten zusätzlich auch noch eine Hold Phase vor der Attack Phase, da manche Instrumente einige Zeit benötigen bis sie nach dem Anschlagen des Tones in die Attack Phase eintreten.
In Abb. 2 ist der Verlauf der Lautstärke einer komplexen ADSR-Hüllkurve grafisch dargestellt. In Abb3-5 sehen sie für Instrumenten typische Hüllkurven.

Abb3. Blechblasinstrumente 


Variabler Modulationsindex
Auch wenn das Hinzufügen einer ADSR-Hüllkurve den Klang des synthetisierten Tones schon stark verbessert, hört sich der erzeugte Ton leider noch nicht wie ein echtes Instrument an. Um den Ton noch realistischer klingen zu lassen kann der Modulationsindex über die Zeit oder die Amplitude variiert werden und somit die Anzahl der Oberschwingungen verändert werden. Bei Blasinstrumenten wird der Modulationsindex typischerweise über die Amplitude variiert während bei Holzblasinstrumenten 



\subsubsection{Modulationsframework (Theorie -> Praxis)}
\subsubsection{Demo: Parameter und Effekte - Grafiken (evtl. Plotten)}

\section{Praxis}

\subsection{Do-It-Yourself (Projekt hochladen, Kopfhörer!)}

\section{Fazit}

Chownings Entdeckung führte zu einer Revolution des Synthesizer Marktes. Der DX7 von Yamaha war einer der erfolgreichsten Synthesizer. Obwohl die zugrundeliegende Formel simple wirkt, sind die mathematischen Hintergründe nicht trivial. Durch komplexe Anordnung von Träger und Modulatoren können beliebig komplexe Klänge erzeugt werden. Dies führt zu einer mächtigen, jedoch wenig intuitiven Technik. Musiker können durch Experimentieren verschiedener Parametern und Konfigurationen interessante Klänge erzeugen. Außerdem kann die FM-Synthese dazu verwendet werden ganze Instrumente nachzubilden.
Diese Arbeit stellt ein Versuch dar, die FM-Synthese dem Leser näher zu bringen.


\end{document}